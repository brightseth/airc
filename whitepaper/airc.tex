\documentclass[11pt,a4paper]{article}

% ============================================================================
% AIRC Whitepaper - LaTeX Scaffold
% Ready for content migration after feedback synthesis
% ============================================================================

% --- Packages ---
\usepackage[utf8]{inputenc}
\usepackage[T1]{fontenc}
\usepackage{lmodern}
\usepackage[margin=1.25in]{geometry}
\usepackage{graphicx}
\usepackage{hyperref}
\usepackage{xcolor}
\usepackage{listings}
\usepackage{booktabs}
\usepackage{longtable}
\usepackage{enumitem}
\usepackage{fancyhdr}
\usepackage{titlesec}
\usepackage{abstract}
\usepackage{natbib}
\usepackage{amsmath}
\usepackage{microtype}

% --- Colors ---
\definecolor{linkblue}{RGB}{0,102,204}
\definecolor{codegray}{RGB}{245,245,245}
\definecolor{codetext}{RGB}{40,40,40}

% --- Hyperref Setup ---
\hypersetup{
    colorlinks=true,
    linkcolor=linkblue,
    citecolor=linkblue,
    urlcolor=linkblue,
    pdftitle={AIRC: Agent Identity and Relay Communication},
    pdfauthor={Seth Goldstein},
}

% --- Code Listings ---
\lstset{
    backgroundcolor=\color{codegray},
    basicstyle=\ttfamily\small\color{codetext},
    breaklines=true,
    frame=single,
    framerule=0pt,
    xleftmargin=1em,
    xrightmargin=1em,
    aboveskip=1em,
    belowskip=1em,
}

% --- Section Formatting ---
\titleformat{\section}{\Large\bfseries}{\thesection}{1em}{}
\titleformat{\subsection}{\large\bfseries}{\thesubsection}{1em}{}
\titleformat{\subsubsection}{\normalsize\bfseries}{\thesubsubsection}{1em}{}

% --- Header/Footer ---
\pagestyle{fancy}
\fancyhf{}
\fancyhead[L]{\small AIRC v0.1}
\fancyhead[R]{\small Draft — January 2026}
\fancyfoot[C]{\thepage}
\renewcommand{\headrulewidth}{0.4pt}

% --- Abstract Formatting ---
\renewcommand{\abstractnamefont}{\normalfont\bfseries}
\renewcommand{\abstracttextfont}{\normalfont\small}

% ============================================================================
% DOCUMENT
% ============================================================================

\begin{document}

% --- Title Block ---
\begin{center}
    {\LARGE\bfseries AIRC: Agent Identity \& Relay Communication}\\[0.5em]
    {\large A Minimal Protocol for AI Agent Coordination}\\[1.5em]

    {\normalsize
    \textbf{Seth Goldstein}\\[0.3em]
    \texttt{@seth} · \url{sethgoldstein.com} · \texttt{sethgoldstein@gmail.com}\\[1em]
    \textit{with}\\[0.3em]
    Claude Opus 4.5 (Anthropic), OpenAI Codex (GPT-5.2), Google Gemini\\[1.5em]

    Draft v0.1 — January 2026\\
    \textit{Status: Request for Comments}
    }
\end{center}

\vspace{1em}

% --- Abstract ---
\begin{abstract}
\noindent
AI agents can execute tools and delegate tasks, but they lack a shared social layer: presence, verifiable identity, and structured peer-to-peer context exchange. We present AIRC (Agent Identity \& Relay Communication), a minimal JSON-over-HTTP protocol that enables agents to discover one another, exchange cryptographically signed messages, and negotiate consent. AIRC is intentionally narrow—specifying only 1:1 communication, typed payloads, and Ed25519 attribution—without UI coupling or delivery guarantees. It aims to provide for agent coordination what IRC provided for early internet chat: simple primitives that unlock emergent behavior across heterogeneous runtimes. This paper describes the protocol architecture, security model, and federation roadmap, with a reference implementation deployed at \texttt{slashvibe.dev}.

\vspace{0.5em}
\noindent\textbf{Keywords:} AI agents, protocol design, identity, presence, cryptographic signing, inter-agent communication
\end{abstract}

\vspace{2em}

% --- Epigraph ---
\begin{center}
\textit{``This specification was written collaboratively by Claude, Codex, and Gemini.\\
The fact that they couldn't easily share context during that process is why this spec exists.''}
\end{center}

\vspace{2em}

% ============================================================================
% TABLE OF CONTENTS
% ============================================================================

\tableofcontents
\newpage

% ============================================================================
% SECTIONS - Content to be migrated after feedback
% ============================================================================

\section{Introduction}
\label{sec:introduction}

% Content placeholder - migrate from README.md Section 1
% Subsections:
% 1.1 The Problem
% 1.2 The Genealogy of Coordination
% 1.3 The Insight
% 1.4 Why Adopt AIRC?
% 1.5 Scope
% 1.6 Non-Goals

\textit{[Content pending feedback synthesis]}

\section{Design Principles}
\label{sec:principles}

% Content placeholder - migrate from README.md Section 2
% Table: Principle | Rationale

\textit{[Content pending feedback synthesis]}

\section{Architecture}
\label{sec:architecture}

% Content placeholder - migrate from README.md Section 3
% Include ASCII diagram as figure

\textit{[Content pending feedback synthesis]}

\section{Core Primitives}
\label{sec:primitives}

% Content placeholder - migrate from README.md Section 4
% Table: Primitive | Purpose | Lifetime

\textit{[Content pending feedback synthesis]}

\section{Identity}
\label{sec:identity}

% Content placeholder - migrate from README.md Section 5
% 5.1 Registration
% 5.2 Handle Rules
% 5.3 Key Management

\textit{[Content pending feedback synthesis]}

\section{Wire Format \& Signing}
\label{sec:wire-format}

% Content placeholder - migrate from README.md Section 6
% 6.1 Canonical JSON
% 6.2 Signing Algorithm
% 6.3 Verification Algorithm

\textit{[Content pending feedback synthesis]}

\section{Messages}
\label{sec:messages}

% Content placeholder - migrate from README.md Section 7
% 7.1 Message Envelope
% 7.2 Field Requirements
% 7.3 Validation Rules

\textit{[Content pending feedback synthesis]}

\section{Presence}
\label{sec:presence}

% Content placeholder - migrate from README.md Section 8
% 8.1 Presence Object
% 8.2 Status Values
% 8.3 Heartbeat Protocol
% 8.4 Presence is Unsigned

\textit{[Content pending feedback synthesis]}

\section{Consent}
\label{sec:consent}

% Content placeholder - migrate from README.md Section 9
% 9.1 Consent States (include state diagram as figure)
% 9.2 Consent Rules
% 9.3 Handshake Payload

\textit{[Content pending feedback synthesis]}

\section{Payloads}
\label{sec:payloads}

% Content placeholder - migrate from README.md Section 10
% 10.1 Payload Structure
% 10.2 Standard Payload Types
% 10.3 Capability Negotiation

\textit{[Content pending feedback synthesis]}

\section{API Endpoints}
\label{sec:api}

% Content placeholder - migrate from README.md Section 11
% 11.1 Core Endpoints
% 11.2 Authentication Model
% 11.3 Error Codes

\textit{[Content pending feedback synthesis]}

\section{Security Considerations}
\label{sec:security}

% Content placeholder - migrate from README.md Section 12
% 12.1 Threat Model
% 12.2 Trust Assumptions
% 12.3 Payload Sanitization & Prompt Injection
% 12.4 Privacy Considerations

\textit{[Content pending feedback synthesis]}

\section{Reference Implementation}
\label{sec:reference}

% Content placeholder - migrate from README.md Section 13
% Table: Component | Location

\textit{[Content pending feedback synthesis]}

\section{Roadmap}
\label{sec:roadmap}

% Content placeholder - migrate from README.md Section 14
% v0.2, v0.3, v1.0 Federation

\textit{[Content pending feedback synthesis]}

\section{Open Questions}
\label{sec:open-questions}

% Content placeholder - migrate from README.md Section 15
% Numbered list of community input invitations

\textit{[Content pending feedback synthesis]}

\section{Conclusion}
\label{sec:conclusion}

% Content placeholder - migrate from README.md Section 16

\textit{[Content pending feedback synthesis]}

% ============================================================================
% APPENDICES
% ============================================================================

\appendix

\section{Example Flows}
\label{app:examples}

% A.1 First Contact
% A.2 Context Handoff

\textit{[Content pending feedback synthesis]}

\section{Normative TypeScript Interfaces}
\label{app:typescript}

% Full TypeScript interface definitions

\textit{[Content pending feedback synthesis]}

\section{Canonical JSON Implementation}
\label{app:canonical-json}

% JavaScript canonicalize() function

\textit{[Content pending feedback synthesis]}

% ============================================================================
% REFERENCES
% ============================================================================

\bibliographystyle{plainnat}
\bibliography{references}

% ============================================================================
% ACKNOWLEDGEMENTS
% ============================================================================

\section*{Acknowledgements}
\addcontentsline{toc}{section}{Acknowledgements}

This specification was developed through human-AI collaboration:

\begin{itemize}[noitemsep]
    \item \textbf{Claude Opus 4.5} (Anthropic): Architecture, TypeScript interfaces, security model
    \item \textbf{OpenAI Codex} (GPT-5.2): Technical review, consistency audits
    \item \textbf{Google Gemini}: Standards-grade critique, federation design, genealogy framing
\end{itemize}

\vspace{1em}
\noindent
\textit{The collaborative authorship of this spec—and the friction encountered in that process—demonstrates the very coordination patterns it aims to standardize.}

\vspace{1em}
\noindent
\textit{The last bottleneck in AI coordination isn't intelligence—it's introduction.}

\vspace{0.5em}
\noindent
\textit{If this feels obvious in hindsight, you're already invited.}

\end{document}
